\section{Baselines}
\subsection{Получение baseline-ов}

Попробуем обучить основные классификаторы на полученных данных. Рассмотрим такие классификаторы, как GradientBoostingClassifier, RandomForestClassifier, LogisticRegression, LinearSVC, без подбора параметров. Приведем результаты для всех датасетов (в случае BSNLP воспользуемся данными для обучения FactRuEval и приведем предсказанные имена классов к именам BSNLP по правилам LocOrg ${\Rightarrow}$ LOC, Other ${\Rightarrow}$ MISC):


\begin{table}[ht]
\centering
\caption{CoNLL2003 Results}
\label{conll2003}
\begin{tabular}{|l|l|l|l|l|l|}
\hline
            & ORG   & LOC   & MISC  & PER   & Total \\ \hline
LogReg      & 0.760 & 0.820 & 0.787 & 0.888 & 0.826 \\ \hline
RF          & 0.676 & 0.725 & 0.682 & 0.849 & 0.752 \\ \hline
LinSVC      & 0.777 & 0.823 & 0.799 & 0.887 & 0.832 \\ \hline
GB          & 0.677 & 0.729 & 0.748 & 0.834 & 0.758 \\ \hline
\end{tabular}
\end{table}



\begin{table}[ht]
\centering
\caption{FactRuEval Results}
\label{factrueval}
\begin{tabular}{|l|l|l|l|l|l|}
\hline
          & Per    &      Loc & Org   & LO     & Total \\ \hline
LogReg    & 0.802  & 0.551    & 0.460 & 0.555  & 0.577 \\ \hline
RF        & 0.763  & 0.475    & 0.364 & 0.460  & 0.501 \\ \hline
LinSVC    & 0.804  & 0.553    & 0.512 & 0.559  & 0.603 \\ \hline
GB        & 0.759  & 0.493    & 0.413 & 0.509  & 0.530 \\ \hline
\end{tabular}
\end{table}



\begin{table}[ht]
\centering
\caption{BSNLP EU Results}
\label{bsnlp_eu}
\begin{tabular}{|l|l|l|l|l|l|}
\hline
            & ORG   & LOC   & MISC  & PER   & Total \\ \hline
LogReg      & 0.652 & 0.504 & 0.000 & 0.313 & 0.525 \\ \hline
RF          & 0.386 & 0.403 & 0.000 & 0.225 & 0.328 \\ \hline
LinSVC      & 0.668 & 0.540 & 0.000 & 0.359 & 0.543 \\ \hline
GB          & 0.655 & 0.497 & 0.000 & 0.368 & 0.529 \\ \hline
\end{tabular}
\end{table}



\begin{table}[ht]
\centering
\caption{BSNLP Trump Results}
\label{bsnlp_trump}
\begin{tabular}{|l|l|l|l|l|l|}
\hline
            & ORG   & LOC   & MISC  & PER   & Total \\ \hline
LogReg      & 0.426 & 0.820 & 0.000 & 0.883 & 0.756 \\ \hline
RF          & 0.289 & 0.700 & 0.000 & 0.823 & 0.674 \\ \hline
LinSVC      & 0.391 & 0.805 & 0.000 & 0.860 & 0.734 \\ \hline
GB          & 0.285 & 0.782 & 0.000 & 0.840 & 0.703 \\ \hline
\end{tabular}
\end{table}

\subsection{Оценка результатов}

Оценка результатов работает с проверкой полного совпадения предсказанной сущности с истинной, если же есть нарушение - вся сущность не попадает в TP, подобный метод проверки отличается от потокеновой проверки (потокеновая проверка более "мягкая"). Для получения итогового результата (не разбитого по классам) складываем полученные по классам TP, FP, FN и вычисляем F1 меру.