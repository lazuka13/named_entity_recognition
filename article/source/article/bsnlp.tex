\section{BSNLP}

BSNLP (Balto-Slavic Natural Language Processing) - конференция по языконезависимой обработке естественного языка, в котором участники работают с славянскими и балтийскими языками - (Croatian, Czech, Polish, Russian, and Slovene, slovak, Ukrainian). В 2017 году общей задачей конференции стало выделение именованных сущностей, их нормализация и межязыковое связывание.

\subsection{Описание корпуса}

Организаторами были подготовлены два датасета, первый содержит документы, относящиеся к Дональду Трампу, текущему президенту США, а второй - документы, упоминающие Европейскую Комиссию. Документы для датасетов были созданы следующим образом: для каждой из тем были произведены поисковые запросы в Google на каждом из семи языков, результаты запроса были очищены от дубликатов и обработаны HTML парсером для извлечения текста (большинство ресурсов были новостями или фрагментами их). Полученный набор частично очищенных документов был использован для отбора 20-25 документов для каждого языка и темы для подготовки финального тестового датасета. Аннотации в основном были сделаны носителями языков, межязыковое связывание - носителями двух языков. \\


Организаторы не предоставляли данных для обучения алгоритмов, участники были вынуждены решать этот вопрос самостоятельно. Входные данные были представлены в следующем формате:

{\tt \small
\begin{verbatim}
<DOCUMENT-ID>
<LANGUAGE>
<CREATION-DATE>
<URL>
<TITLE>
<TEXT>
\end{verbatim}
}
В качестве выходных данных от участников ожидались документы с выделенными и нормализованными именованными сущностями, указанием их типов и межъязыковых идентификаторов:
{\tt \small
\begin{verbatim}
<DOCUMENT-ID>
<MENTION> TAB <BASE> TAB <CAT> TAB <ID>

16
Podlascy Czeczeni Podlascy Czeczeni PER 1
ISIS ISIS ORG 2
Rosji Rosja LOC 3
Rosja Rosja LOC 3
Polsce Polska LOC 4
Warszawie Warszawa LOC 5
Magazynu Kuriera Porannego Magazyn Kuriera\
Porannego ORG 6
\end{verbatim}
}

\subsection{Итоги соревнования}

В соревновании приняли участие более 11 команд, но только 2 из них сумели предоставить свое решение в поставленный организаторами срок. В связи с этим соревнование было продлено, и на данный момент сайт конференции содержит информацию о четырех различных системах: JHU, Liner2, LexiFlexi, Sharoff.
Приведем краткое описание этих систем и достигнутых ими результатов.

\subsubsection{JHU}

Авторы системы JHU приняли участие только в задачах по выделению и межъязыковому связыванию именованных сущностей. Для создания своей модели они проделали следующие шаги:

\begin{enumerate}
    \item Получили из публично доступных датасетов параллельные тексты для языков соревнования и английского
    \item Применили к текстам на английском уже готовую модель выделения именованных сущностей (Illinois Named Entity Tagger)
    \item Спроецировали полученные результаты с английских текстов на целевые языки при помощи  Giza++
    \item На полученных датасетах обучили SVMLattice named entity recognizer
    \item Использовали систему Kripke для межъязыкового связывания
\end{enumerate}

Система показала хорошие результаты во всех трех задачах соревнования, заняв первые - вторые места практически во всех языках.

\begin{table}[ht]
\centering
\caption{F1 scores by type and language}
\label{JHU_trump_categories}
\begin{tabular}{|l|l|l|l|l|}
\hline
    & PER   & ORG   & LOC   & MISC \\ \hline
ces & 53.30 & 21.77 & 68.12 & 0.00 \\ \hline
hrv & 60.10 & 29.36 & 63.19 & 3.39 \\ \hline
pol & 35.29 & 13.19 & 68.73 & 0.00 \\ \hline
rus & 41.77 & 14.55 & 65.03 & 0.00 \\ \hline
slk & 57.52 & 18.67 & 63.20 & 2.94 \\ \hline
slv & 55.92 & 18.18 & 65.63 & 0.00 \\ \hline
ukr & 29.56 & 6.45  & 56.83 & 0.00 \\ \hline
all & 49.26 & 18.16 & 64.80 & 1.08 \\ \hline
\end{tabular}
\end{table}

\subsubsection{Liner2}

Авторы модели использовали фреймворк Liner2, который предоставляет набор модулей, основанных на статистических моделях, словарях, правилах и эвристиках и аннотирующих различные типа фраз. Команда работала только с польским языком, и сумела достичь лучших результатов в задачах выделения и нормализации именованных сущностей. Результаты работы системы можно увидеть ниже в таблице.

\begin{table}[ht]
\centering
\caption{Liner2 Results}
\label{liner2}
\begin{tabular}{|l|l|l|l|}
\hline
Task                 & P     & R     & F     \\ \hline
Names matching       &       &       &       \\ \hline
Relaxed partial      & 66.24 & 63.27 & 64.72 \\ \hline
Relaxed exact        & 65.40 & 62.78 & 64.07 \\ \hline
Strict               & 71.10 & 58.81 & 66.61 \\ \hline
Normalization        & 75.50 & 44.44 & 55.95 \\ \hline
Coreference          &       &       &       \\ \hline
Document level       & 7.90  & 42.71 & 12.01 \\ \hline
Language level       & 3.70  & 8.00  & 5.05  \\ \hline
Cross-language level & n/a   & n/a   & n/a   \\ \hline
\end{tabular}
\end{table}

\subsubsection{LexiFlexi}

К сожалению, авторы системы не предоставили статьи о своей системе, поэтому приходится довольствоваться ее кратким описанием. LexiFlexi применяет 3 лексико-семантических ресурса ко входному тексту в следующем порядке:

\begin{enumerate}
    \item Сопоставляет имена из базы данных JRC Variant Names 
    \item Сопоставляет имена из огромной коллекции названий сущностей на различных языках полу-автоматически полученной из применения BabelNet к еще не обработанному тексту
    \item Сопоставляет топонимы из газетира GeoNames в еще не обработанном тексте
\end{enumerate}

В конце работы системы применяются несколько языко-независимых эвристик для нахождения вариантов (аббревиатур) именованных сущностей, распознанных на предыдущих шагах.

Данная модель показала средние результаты в задачах выделения и нормализации сущностей, однако заняла практически все первые места в задаче межъязыкового связывания на уровне документов, и половину первых мест - на уровне языка.

\subsubsection{Sharoff}

Система Сергея Шарова - пример применения метода адаптации языка к задаче выделения именованных сущностей. Автор модели создал мультиязычное пространство embedding-ов для слов, основываясь на модели Dinu [] с добавлением взвешенного расстояния Левенштейна. Это пространство было использовано для обучения NER таггера, созданного при помощи нейронной сети, базируясь на Словенском NER корпусе. Простыми словами, главная идея - если мы можем адаптировать модель отзывов к фильмам для работы с отзывами к отелями, то наверное мы можем адаптировать модель для NER одного языка к работе с родственными языками.

Модель добилась неплохих результатов в задачах выделения и нормализации именных сущностей, заняла первые места на чешском и словенском языках, заметны также и сильные падения качества на русском и украинском. На задаче межъязыкового связывания система проявила себя не очень хорошо, везде осталась на последних местах. 

\subsection{Выводы}

На мой взгляд, на данной конференции были показаны две интересные системы - система Сергея Шарова и система JHU. Система Liner2 была применена только к польскому языку, а система LexiFlexi, если верить ее описанию, просто работает с большим набором газетиров. Результаты, достигнутые системой Шарова и системой JHU, показывают применимость методов адаптации языка и параллельной обработки текстов к задачам NER.