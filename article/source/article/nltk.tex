\section{Настройка окружения, пакет NLTK, корпуса}


Исследовательская работа проводилась в ОС Windows 10, использовалась среда для научных исследований Anaconda, Python3. Дополнительно установил пакет pymorphy2 для POS-тегирования токенов из датасетов на русском языке, для английского датасета (CoNLL2003) использовались предоставленные организаторами теги (chunk и POS). Для удобной работы с датасетами они были приведены к модифицированному формату датасетов CoNLL (были добавлены столбцы OFFSET и LEN - отступ токена и его длина), в коде они хранятся в виде модифицированных NLTK Corpus-ов. (код обработки можно найти в файлах .ipynb, а код NLTK Corpus-а - в файле corpus.py)