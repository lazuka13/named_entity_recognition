\documentclass[letterpaper,twocolumn,10pt]{article}
\usepackage[T2A]{fontenc}
\usepackage[utf8]{inputenc}
\usepackage[russian]{babel}
\usepackage{styles,epsfig,endnotes}
\usepackage{listings}

\begin{document}

\date{}

\title{\Large \textbf{Named Entity Recognition на лексических признаках с учётом всех вхождений упоминания в текст} }

\author{
{\rm Латыпов Зуфар}\\
Московский Физико-Технический Институт
}

\maketitle

\thispagestyle{empty}

\subsection*{Abstract}
В данной статье рассматривается возможности использования различных глобальных признаков для задачи выделения именованных сущностей, проводится обзор используемых ранее и современных методов решения этой задачи, а также приводится алгоритм, использованный для получения результатов на трех датасетах - FactRuEval2016, BSNLP2017, CoNLL2003.

\section{CoNLL 2003}

CoNLL 2003 (Conference on Computational Natural Language Learning) - конференция по машинной обработке естественного языка, прошедшая в Канаде в 2003 году. Общей задачей конференции было решение проблемы NER, Распознавания Именованных Сущностей, для двух языков - немецкого и английского. Для измерения точности использовались метрики точности (precision), полноты (recall) и F-мера (F-measure), участие приняли 16 различных систем, наилучшим результом стали 88.76 для английского и 72.41 для немецкого от системы FIJZ03 (здесь и в дальнейшем результаты указаны по метрике F1, если не указано обратное, кроме того, данные результаты вычислялись путем усреднения качества по всем типам сущностей). Ниже в таблицах 1 и 2 приведем также качество в разбивку по точности/полноте/F-мере (5 наилучших результатов для каждого языка):

\begin{table}[ht]
\caption{Первый датасет}
\centering
\label{first_dataset}
\begin{tabular}{|l|l|l|l|}
\hline
English      & Precision & Recall  & F-measure \\ \hline
FIJZ03       & 88.99\%   & 88.54\% & 88.76${\pm}$0.7 \\ \hline
CN03         & 88.12\%   & 88.51\% & 88.31${\pm}$0.7 \\ \hline
KSNM03       & 85.93\%   & 86.21\% & 86.07${\pm}$0.8 \\ \hline
ZJ03         & 86.13\%   & 84.88\% & 85.50${\pm}$0.9 \\ \hline
CMP03b       & 84.05\%   & 85.96\% & 85.00${\pm}$0.8 \\ \hline
baseline     & 71.91\%   & 50.90\% & 59.61${\pm}$1.2 \\ \hline
\end{tabular}
\end{table}

\begin{table}[ht]
\caption{Второй датасет}
\centering
\label{second_dataset}
\begin{tabular}{|l|l|l|l|}
\hline
German      & Precision & Recall  & F-measure \\ \hline
FIJZ03      & 83.87\%   & 63.71\% & 72.41${\pm}$1.3 \\ \hline
KSNM03      & 80.38\%   & 65.04\% & 71.90${\pm}$1.2 \\ \hline
ZJ03        & 82.00\%   & 63.03\% & 71.27${\pm}$1.5 \\ \hline
MMP03       & 75.97\%   & 64.82\% & 69.96${\pm}$1.4 \\ \hline
CMP03b      & 75.47\%   & 63.82\% & 69.15${\pm}$1.3 \\ \hline
baseline    & 31.86\%   & 28.89\% & 30.30${\pm}$1.3 \\ \hline
\end{tabular}
\end{table}

\subsection{Описание корпуса}

Датасет состоит из 6 файлов - это файлы testa, testb, train для каждого из языков. testa использовался для проверки модели при разработке, testb - для итоговой оценки модели. Файлы содержат 4 столбца, разделенные пробелами. Первый элемент каждой строки - это само слово, второй - POS (part-of-speach) tag, третий - syntactic chunk tag, четвертый - named entity tag. Также интересной особенностью 2003 года стали предоставленные списки именованных сущностей и неразмеченные данные, которые предлагалось как-то использовать для улучшения системы. Английский корпус был представлен коллекцией новостных статей из Reuters Corpus. Аннотация была произведена в University of Antwerp. Немецкий корпус - коллекция статей от Frankfurter Rundschau. Пример содержимого файла train, а также сводная таблица (3) по размерам датасетов приведены ниже.

{\tt \small
\begin{verbatim}
WORD         POS  CHUNK NE
U.N.         NNP  I-NP  I-ORG 
official     NN   I-NP  O 
Ekeus        NNP  I-NP  I-PER 
heads        VBZ  I-VP  O 
for          IN   I-PP  O 
Baghdad      NNP  I-NP  I-LOC 
.            .    O     O 
\end{verbatim}
}


\begin{table}[ht]
\centering
\caption{Размеры датасетов}
\label{size}
\begin{tabular}{|l|l|l|l|}
\hline
\textbf{} & \textbf{Learning} & \textbf{Validating} & \textbf{Testing} \\ \hline
Articles  & 946               & 216                 & 231              \\ \hline
Sentences & 14987             & 3466                & 3684             \\ \hline
Tokens    & 203621            & 51362               & 46435            \\ \hline
LOC       & 7140              & 1837                & 1668             \\ \hline
MISC      & 3438              & 922                 & 702              \\ \hline
ORG       & 6321              & 1341                & 1661             \\ \hline
PER       & 6600              & 1842                & 1617             \\ \hline
\end{tabular}
\end{table}

\subsection{Итоги дорожки}

Большинство систем на английском языке показали результаты в районе 88 - 80 при baseline в 71. На немецком языке системы проявили себя хуже - максимум 72, большинство работ в районе 60-70, однако и baseline тут значительно ниже - 30. Рассмотрим трех участников, показавших лучшие результаты для английского языка:

\subsubsection{FIJZ03 - 88.76}

Первое место заняла модель команды FIJZ03, достигшая результата в 88.76 \cite{Florian:2003:NER:1119176.1119201}. Авторы модели использовали комбинацию четырех различных классификаторов - линейный классификатор, максимальной энтропии, основанное на трансформации обучение и скрытую Марковскую модель. Без газетиров и других дополнительных ресурсов они достигли результата в 91.6 на тренировочных данных, с использованием дополнительных данных сумели получить дополнительное уменьшение ошибки на 15 - 20 процентов. Также авторы отмечают, что устойчивый классификатор минимизации риска "выглядит особенно подходящим для обработки дополнительных источников признаков, и потому является хорошим кандидатом для комбинации классификаторов". Результаты работы данной модели приведены в таблицах 4 и 5.\\


Список рассматриваемых признаков:
\begin{itemize}
\setlength\itemsep{0em}
\item слова и их леммы в окне размеров в пять слов около текущего
\item POS тег текущего и окружающего слов
\item текстовые чанки в окне -1..1
\item префиксы и суффиксы длины до 4 букв текущего и окружающего слов
\item флаги, отражающие наличие заглавных букв (firstCap, 2digit and allCaps)
\item информация из газетира
\item результат работы двух других классификаторов, натренированных на более богатом датасете с большим числом категория
\end{itemize}

\begin{table}[ht]
\centering
\caption{FIJZ03 English Test}
\label{1place_eng}
\begin{tabular}{|l|l|l|l|}
\hline
\textbf{English test} & \textbf{Precision} & \textbf{Recall} & \textbf{F1} \\ \hline
LOC                   & 90.59\%            & 91.73\%         & 91.15       \\ \hline
MISC                  & 83.46\%            & 77.64\%         & 80.44       \\ \hline
ORG                   & 85.93\%            & 83.44\%         & 84.67       \\ \hline
PER                   & 92.49\%            & 95.24\%         & 93.85       \\ \hline
overall               & 88.99\%            & 88.54\%         & 88.76       \\ \hline
\end{tabular}
\end{table}

\begin{table}[ht]
\centering
\caption{FIJZ03 German Test}
\label{1place_ger}
\begin{tabular}{|l|l|l|l|}
\hline
\textbf{German test} & \textbf{Precision} & \textbf{Recall} & \textbf{F1} \\ \hline
LOC                  & 80.19\%            & 71.59\%         & 75.65       \\ \hline
MISC                 & 77.87\%            & 41.49\%         & 54.14       \\ \hline
ORG                  & 79.43\%            & 54.46\%         & 64.62       \\ \hline
PER                  & 91.93\%            & 75.31\%         & 82.80       \\ \hline
overall              & 83.87\%            & 63.71\%         & 72.41       \\ \hline
\end{tabular}
\end{table}

\subsubsection{CN03 - 88.31}

Авторы модели использовали подход, основанный на принципе максимума энтропии, причем использовали в качестве признаков не только локальный контекст, но также использовали и остальные вхождения этого слова для извлечения полезных признаков (т.н. глобальные признаки) \cite{Chieu:2003:NER:1119176.1119199}. Для этого они обработали датасет и создали несколько списков слов - Frequent Word List, Useful Unigrams, Useful Bigrams, Useful Word Suffixes, Useful Name Class Suffixes, Function Words, которые в дальнейшем использовались для выделения глобальных признаков. \\

К сожалению, в предоставленной авторами статье нет никакой информации по отбору признаков, только их перечисление, а название Useful, например, Useful Unigrams, говорит о том, что лист содержит 1-граммы, которые часто предшествуют определенному типу сущностей, а потому могут быть полезны при классификации. Однако в статье довольно подробно описаны листы и получаемые из них признаки, так что она может послужить основой для дальнейшего изучения возможностей по использованию глобальных признаков. Результаты работы данной модели приведены в таблицах 6 и 7.

\begin{table}[ht]
\centering
\caption{CN03 English Test}
\label{2place_eng}
\begin{tabular}{|l|l|l|l|}
\hline
\textbf{English test} & \textbf{Precision} & \textbf{Recall} & \textbf{F1} \\ \hline
LOC                   & 90.88\%            & 91.37\%         & 91.12       \\ \hline
MISC                  & 80.15\%            & 78.21\%         & 79.16       \\ \hline
ORG                   & 83.82\%            & 84.83\%         & 84.32       \\ \hline
PER                   & 93.07\%            & 93.82\%         & 93.44       \\ \hline
Overall               & 88.12\%            & 88.51\%         & 88.31       \\ \hline
\end{tabular}
\end{table}

\begin{table}[ht]
\centering
\caption{CN03 German Test}
\label{2place_ger}
\begin{tabular}{|l|l|l|l|}
\hline
\textbf{German test} & \textbf{Precision} & \textbf{Recall} & \textbf{F1} \\ \hline
LOC                  & 69.23\%            & 59.13\%         & 63.78       \\ \hline
MISC                 & 62.05\%            & 33.43\%         & 43.45       \\ \hline
ORG                  & 76.70\%            & 48.12\%         & 59.14       \\ \hline
PER                  & 88.82\%            & 75.15\%         & 81.41       \\ \hline
Overall              & 76.83\%            & 57.34\%         & 65.67       \\ \hline
\end{tabular}
\end{table}


\subsubsection{KSNM03 - 86.07}

Авторы рассматривают две модели - скрытую марковскую модель и conditional markov model, рассматривая в качестве базовых единиц не слова, а символы и n-граммы \cite{Klein:2003:NER:1119176.1119204}. При разработке первой модели использование контекста было минимально, а при разработке второй использовался подход максимальной энтропии, после чего добавили дополнительные признаки и объединили модели в CMM.
Результаты работы данной модели приведены в таблицах 8 и 9.

\begin{table}[ht]
\centering
\caption{KSNM03 English Test}
\label{3place_eng}
\begin{tabular}{|l|l|l|l|}
\hline
\textbf{English test} & \textbf{Precision} & \textbf{Recall} & \textbf{F1} \\ \hline
LOC                   & 90.04              & 89.93           & 89.98       \\ \hline
MISC                  & 83.49              & 77.07           & 80.15       \\ \hline
ORG                   & 82.49              & 78.57           & 80.48       \\ \hline
PER                   & 86.66              & 95.18           & 90.72       \\ \hline
Overall               & 86.12              & 86.49           & 86.31       \\ \hline
\end{tabular}
\end{table}

\begin{table}[htbp]
\centering
\caption{KSNM03 German Test}
\label{3place_ger}
\begin{tabular}{|l|l|l|l|}
\hline
\textbf{German test} & \textbf{Precision} & \textbf{Recall} & \textbf{F1} \\ \hline
LOC                  & 78.01              & 69.57           & 73.54       \\ \hline
MISC                 & 75.90              & 47.01           & 58.06       \\ \hline
ORG                  & 73.26              & 51.75           & 60.65       \\ \hline
PER                  & 87.68              & 79.83           & 83.57       \\ \hline
Overall              & 80.38              & 65.04           & 71.90       \\ \hline
\end{tabular}
\end{table}

\subsubsection{Итоги}

Подводя итоги, отметим, что в 2003 году часто использовались и хорошо себя проявили такие классификаторы, как HMM и максимальной энтропии. Кроме того, многие авторы отмечали тот факт, что категория MISC довольно сильно повлияла на снижение качества работы моделей, связывая это с обобщенностью данной категории.

\subsection{CRF и современные работы}
\subsubsection{CRF}

Рассмотрим статью Andrew McCallum and Wei Li, в которой они обращаются к CRF, индуцированию признаков и методу WebListing для создания лексиконов \cite{McCallum:2003:ERN:1119176.1119206}. Их система показала неплохой результат в 84.04 (F-мера), что доказывает применимость CRF в задачах выделения именованных сущностей.

\begin{table}[ht]
\centering
\caption{CRF English Test}
\label{crf_eng}
\begin{tabular}{|l|l|l|l|}
\hline
\textbf{English test} & \textbf{Precision} & \textbf{Recall} & \textbf{F1} \\ \hline
LOC                   & 87.23\%            & 87.65\%         & 87.44       \\ \hline
MISC                  & 74.44\%            & 71.37\%         & 72.87       \\ \hline
ORG                   & 79.52\%            & 78.33\%         & 78.92       \\ \hline
PER                   & 91.05\%            & 89.98\%         & 90.51       \\ \hline
Overall               & 84.52\%            & 83.55\%         & 84.04       \\ \hline
\end{tabular}
\end{table}

\begin{table}[ht]
\centering
\caption{CRF German Test}
\label{crf_ger}
\begin{tabular}{|l|l|l|l|}
\hline
\textbf{German test} & \textbf{Precision} & \textbf{Recall} & \textbf{F1} \\ \hline
LOC                  & 71.92\%            & 69.28\%         & 70.57       \\ \hline
MISC                 & 69.59\%            & 42.69\%         & 52.91       \\ \hline
ORG                  & 63.85\%            & 48.90\%         & 55.38       \\ \hline
PER                  & 90.04\%            & 74.14\%         & 81.32       \\ \hline
Overall              & 75.97\%            & 61.72\%         & 68.11       \\ \hline
\end{tabular}
\end{table}

\subsubsection{Современные работы}

В последние годы появилось довольно много статей, рассматривающих использование LSTM-CNNs, LSTM-CRF, LSTM-CNNs-CRF для датасета CoNLL2003. На данный момент один из наилучших результатов (State Of Art) был достигнут в 2016 году Xuezhe Ma и Eduard Hovy, используя BLSTM-CNNs-CRF, они смогли добиться результата в 91.21 (F-мера) без использования сторонних данных \cite{DBLP:journals/corr/MaH16}. В 2015 году была достигнута планка в 91.62 при помощи LSTM-CNNs и использовании двух наборов данных, полученных из публично доступных источников, авторы второй статьи также называют свой результат наилучшим \cite{DBLP:journals/corr/ChiuN15}.  \\

Схема BLSTM-CNNs-CRF:

\begin{enumerate}
\item Используя CNN, извлекают морфологическую информацию, кодируют ее в символьное представление.
\item Отправляют результат первого шага в BLSTM (отмечается важность dropout слоя).
\item Результат работы BLSTM отправляется CRF.
\end{enumerate}

Результатов для каждого из типов сущностей авторы не предоставили.
\section{FactRuEval}

FactRuEval - соревнование по выделению именованных сущностей и извлечению фактов, проведенное на международной конференции по компьютерной лингвистике Диалог. Само соревнование включало в себя 3 дорожки: задачей первой было простое выделение именованных сущностей, определение их типов (персоны, организации и локации, другие не рассматривались) и указание позиции и длины сущности в тексте; для решения второй дорожки нужно было связать все упоминания одной и той же сущности в рамках текста в один объект и определить атрибуты этого объекта; третья задача затрагивала вопрос извлечения фактов из текста, то есть отношений между несколькими объектами.

\subsection{Описание корпуса}

Корпус текстов соревнования состоит из новостных и аналитических текстов на общественно-политическую тему на русском языке. Источниками текстов являются следующие издания: Частный корреспондент, Викиновости, Лентапедия. Корпус разделён на две части: демонстрационную и тестовую. Соотношение количества текстов из разных источников в этих двух частях одинаково. Сбалансированность по каким бы то ни было другим показателям не гарантируется. Работы по разметке этой коллекции текстов были проведены силами добровольцев на сайте OpenCorpora.org под руководством экспертов в областях.

Авторами соревнования была разработана специальная аннотационная модель, которая была использована для аннотации 255 документов (122 текста в обучающей выборке и 133 текста в тестовой). Первые два слоя модели содержат аннотированые упоминания сущностей, третий слой содержит информацию об отношениях кореференции между сущностями, четвертый же слой группирует сущности в факты. Для первой дорожки использовались первые два слоя, для второй - три слоя, для третьей - все четыре слоя.


\begin{table}[ht]
\centering
\caption{Размеры датасета}
\label{factrueval_sizes}
\begin{tabular}{|l|l|l|l|}
\hline
\multicolumn{2}{|l|}{Total texts}  & \multicolumn{2}{l|}{Total characters} \\ \hline
Demo Set         & Test Set        & Demo Set          & TestSet           \\ \hline
122              & 133             & 189893            & 460636            \\ \hline
\multicolumn{2}{|l|}{Total tokens} & \multicolumn{2}{l|}{Total sentences}  \\ \hline
Demo Set         & Test Set        & Demo Set          & Test Set          \\ \hline
30940            & 59382           & 1769              & 3138              \\ \hline
\end{tabular}
\end{table}


Описание слоев:
(Более полное описание слоев может быть найдено в репозитории соревнования) \\

Нулевой слой - это сами токены, без обработки.

{\tt \small
\begin{verbatim}
143783 0 1 В
143784 2 11 понедельник
143785 14 2 28
143786 17 4 июня
143787 22 1 у
143788 24 6 здания
143789 31 5 мэрии
143790 37 6 Москвы
143791 44 2 на
143792 47 8 Тверской
143793 56 7 площади
143794 64 10 состоялась
143795 75 9 очередная
143796 85 19 несанкционированная
\end{verbatim}
}

В первом слое в тексте были выделены типизированные спаны. Это цепочки слов, помеченные одним или более предопределенными тегами. Предполагалось также, что каждый тип выделенного объекта может иметь свой набор тегов, например, в случае упоминания людей различали имена, фамилии, ники. 

{\tt \small
\begin{verbatim}
22763 loc_name 37 6 143790 1  # 143790 Москвы
22764 org_descr 31 5 143789 1  # 143789 мэрии
22765 loc_name 47 8 143792 1  # 143792 Тверской
22766 loc_descr 56 7 143793 1  # 143793 площади
22767 name 313 4 143831 1  # 143831 Юрия
22768 surname 318 7 143832 1  # 143832 Лужкова
\end{verbatim}
}

Во втором слое спаны сгруппированы в типизированные упоминания объектов. 

{\tt \small
\begin{verbatim}
10433 Org 22763 22764 # Москвы мэрии
10547 LocOrg 22763 # Москвы
10434 Location 22765 22766 # Тверской площади
10435 Person 22767 22768 # Юрия Лужкова
\end{verbatim}
}

В третьем слое упоминания объектов, содержащиеся в одном тексте и имеющие одного референта, сгруппированы в идентифицированные объекты. 

{\tt \small
\begin{verbatim}
47 10436 10437 10547
name Москва

48 10435 10441
firstname Юрий
lastname Лужков

49 10433
descriptor мэрия
name мэрия Москвы

50 10434
descriptor площадь
name Тверская площадь

51 10438
name Россия
\end{verbatim}
}

В четвертом слое были выделены факты - типизированные отношения между идентифицированными объектами.

{\tt \small
\begin{verbatim}
100-0 Occupation
Who obj48 Лужков Юрий
Position span22777 мэра
Where obj47 Москва

100-1 Occupation
Who obj168 Громов Борис
Position span22778 губернатора
Where obj637 Подмосковье
\end{verbatim}
}

\subsection{Итоги дорожки}

Большинство систем приняло участие в двух первых дорожках, в решении второй дорожки приняли участие всего 2 команды. Авторы соревнования связывают этот факт с чрезвычайной сложностью и неочевидностью принципа решения проблемы выделения фактов. Статьи были предоставлены только тремя командами, поэтому рассмотрим их.

\subsubsection{Named Entity Recognition in Russian: the Power of Wiki-Based Approach}

Участники команды использовали два различных подхода - на основании только FactRuEval данных и на основании Wiki данных \cite{NOTHMAN2013151}. 

Первый подход - использование широкого набора фичей: аффиксы, сам токен, POS-тег, лемма, предикаты, флаги, характеризующие тот факт, что слово начинается с большой буквы, и другие + Word2Vec из Wiki + словари, построенные на основе Wiki. 

Второй подход - конструируют датасет на основе статей Wiki, и используют его вместо предоставленного. Результаты команды можно увидеть в таблицах ниже.

\begin{table}[ht]
\centering
\caption{Entity Extraction (I)}
\label{factrueval_1_1}
\begin{tabular}{|l|l|l|l|}
\hline
Feature set               & Precision & Recall & F1     \\ \hline
basic                     & 0.7357    & 0.6186 & 0.6720 \\ \hline
basic+dict                & 0.8098    & 0.6988 & 0.7502 \\ \hline
basic+word2vec            & 0.8093    & 0.7241 & 0.7643 \\ \hline
basic+dict+word2vec       & 0.8257    & 0.7408 & 0.7810 \\ \hline
\end{tabular}
\end{table}



\begin{table}[ht]
\centering
\caption{Entity Extraction (по типам)}
\label{factrueval_1_2}
\begin{tabular}{|l|l|l|l|}
\hline
Entity type         & Precision & Recall & F-measure     \\ \hline
Person       & 0.9340    & 0.8675 & 0.8995 \\ \hline
Location     & 0.7259    & 0.6944 & 0.7098 \\ \hline
Organization & 0.7844    & 0.6548 & 0.7137 \\ \hline
LocOrg       & 0.7858    & 0.7251 & 0.7542 \\ \hline
OVERALL      & 0.8257    & 0.7408 & 0.7810 \\ \hline
\end{tabular}
\end{table}



\begin{table}[ht]
\centering
\caption{Entity Extraction (II)}
\label{factrueval_1_3}
\begin{tabular}{|l|l|l|l|l|l|}
\hline
\multicolumn{3}{|l|}{FactRuEval devset} & \multicolumn{3}{l|}{FactRuEval testset} \\ \hline
Precision     & Recall     & F1         & Precision     & Recall     & F1         \\ \hline
0.88        & 0.64    & 0.74     & 0.85        & 0.69     & 0.76     \\ \hline
\end{tabular}
\end{table}

\subsubsection{Named Entity Normalization for Fact Extraction Task}

Использовали rule-based подход, создали свою систему обработки текста, состояющую из токенизатора, морфологического анализатора, газетира, искателя паттернов и извлекателя фактов \cite{FactRuEval2}. 

Модуль, отвечающий за поиск паттернов - основной инструмент для извлечения сущностей, спроектирован для выполнения правил, задаваемых в стиле регулярных выражений.

К сожалению, авторы статьи не указали название своей команды, поэтому приведем результаты, указанные в их статье (предварительных):

\begin{table}[ht]
\centering
\caption{Entity Extraction}
\label{factrueval_2_1}
\begin{tabular}{|l|l|l|l|}
\hline
Entity type   & Precision & Recall & F-measure \\ \hline
Persons       & 0.9300    & 0.8403 & 0.8829    \\ \hline
Locations     & 0.9535    & 0.8361 & 0.8910    \\ \hline
Organizations & 0.8181    & 0.5450 & 0.6542    \\ \hline
OVERALL       & 0.9038    & 0.7301 & 0.8077    \\ \hline
\end{tabular}
\end{table}

\begin{table}[ht]
\centering
\caption{Entity Normalization}
\label{factrueval_2_2}
\begin{tabular}{|l|l|l|l|}
\hline
Entity type   & Precision & Recall & F-measure \\ \hline
Persons       & 0.8024    & 0.8433 & 0.8223    \\ \hline
Locations     & 0.9017    & 0.7741 & 0.8330    \\ \hline
Organizations & 0.6490    & 0.5760 & 0.6103    \\ \hline
OVERALL       & 0.7725    & 0.7173 & 0.7439    \\ \hline
\end{tabular}
\end{table}

\subsubsection{Information Extraction Based on Deep Syntactic-Semantic Analysis}

Команда использовала уже имевшуюся у нее модель, основанную на синтаксическо-семантическом анализе, rule-based подход \cite{FactRuEval3}. Для более подробной информации авторы статьи отсылают читаталей к статьям Анисимовича и Зуева 2012 и 2013 годов с конференции Диалог.
Результаты команды представлены в таблицах ниже:


\begin{table}[ht]
\centering
\caption{Entity Extraction}
\label{factrueval_3_1}
\begin{tabular}{|l|l|l|l|}
\hline
Entity type    & Precision      & Recall      & F-measure     \\ \hline
Persons     & 0.9450 & 0.9155 & 0.9300 \\ \hline
Locations     & 0.9261 & 0.8698 & 0.8971 \\ \hline
Organizations     & 0.8175 & 0.7564 & 0.7858 \\ \hline
OVERALL & 0.8931 & 0.8427 & 0.8672 \\ \hline
\end{tabular}
\end{table}

\begin{table}[ht]
\centering
\caption{Entity Normalization}
\label{factrueval_3_2}
\begin{tabular}{|l|l|l|l|}
\hline
Entity type    & Precision      & Recall      & F-measure     \\ \hline
Persons     & 0.8817 & 0.8592 & 0.8703 \\ \hline
Locations     & 0.8430 & 0.7942 & 0.8179 \\ \hline
Organizations     & 0.6823 & 0.6763 & 0.6793 \\ \hline
OVERALL & 0.7903 & 0.7677 & 0.7789 \\ \hline
\end{tabular}
\end{table}
\section{BSNLP}

BSNLP (Balto-Slavic Natural Language Processing) - конференция по языконезависимой обработке естественного языка, в котором участники работают с славянскими и балтийскими языками - (Croatian, Czech, Polish, Russian, and Slovene, slovak, Ukrainian). В 2017 году общей задачей конференции стало выделение именованных сущностей, их нормализация и межязыковое связывание.

\subsection{Описание корпуса}

Организаторами были подготовлены два датасета, первый содержит документы, относящиеся к Дональду Трампу, текущему президенту США, а второй - документы, упоминающие Европейскую Комиссию. Документы для датасетов были созданы следующим образом: для каждой из тем были произведены поисковые запросы в Google на каждом из семи языков, результаты запроса были очищены от дубликатов и обработаны HTML парсером для извлечения текста (большинство ресурсов были новостями или фрагментами их). Полученный набор частично очищенных документов был использован для отбора 20-25 документов для каждого языка и темы для подготовки финального тестового датасета. Аннотации в основном были сделаны носителями языков, межязыковое связывание - носителями двух языков. \\


Организаторы не предоставляли данных для обучения алгоритмов, участники были вынуждены решать этот вопрос самостоятельно. Входные данные были представлены в следующем формате:

{\tt \small
\begin{verbatim}
<DOCUMENT-ID>
<LANGUAGE>
<CREATION-DATE>
<URL>
<TITLE>
<TEXT>
\end{verbatim}
}
В качестве выходных данных от участников ожидались документы с выделенными и нормализованными именованными сущностями, указанием их типов и межъязыковых идентификаторов:
{\tt \small
\begin{verbatim}
<DOCUMENT-ID>
<MENTION> TAB <BASE> TAB <CAT> TAB <ID>

16
Podlascy Czeczeni Podlascy Czeczeni PER 1
ISIS ISIS ORG 2
Rosji Rosja LOC 3
Rosja Rosja LOC 3
Polsce Polska LOC 4
Warszawie Warszawa LOC 5
Magazynu Kuriera Porannego Magazyn Kuriera\
Porannego ORG 6
\end{verbatim}
}

\subsection{Итоги соревнования}

В соревновании приняли участие более 11 команд, но только 2 из них сумели предоставить свое решение в поставленный организаторами срок. В связи с этим соревнование было продлено, и на данный момент сайт конференции содержит информацию о четырех различных системах: JHU, Liner2, LexiFlexi, Sharoff.
Приведем краткое описание этих систем и достигнутых ими результатов.

\subsubsection{JHU}

Авторы системы JHU приняли участие только в задачах по выделению и межъязыковому связыванию именованных сущностей \cite{mayfield-mcnamee-costello:2017:BSNLP}. Для создания своей модели они проделали следующие шаги:

\begin{enumerate}
    \item Получили из публично доступных датасетов параллельные тексты для языков соревнования и английского
    \item Применили к текстам на английском уже готовую модель выделения именованных сущностей (Illinois Named Entity Tagger)
    \item Спроецировали полученные результаты с английских текстов на целевые языки при помощи  Giza++
    \item На полученных датасетах обучили SVMLattice named entity recognizer
    \item Использовали систему Kripke для межъязыкового связывания
\end{enumerate}

Система показала хорошие результаты во всех трех задачах соревнования, заняв первые - вторые места практически во всех языках.

\begin{table}[ht]
\centering
\caption{F1 scores by type and language}
\label{JHU_trump_categories}
\begin{tabular}{|l|l|l|l|l|}
\hline
    & PER   & ORG   & LOC   & MISC \\ \hline
ces & 53.30 & 21.77 & 68.12 & 0.00 \\ \hline
hrv & 60.10 & 29.36 & 63.19 & 3.39 \\ \hline
pol & 35.29 & 13.19 & 68.73 & 0.00 \\ \hline
rus & 41.77 & 14.55 & 65.03 & 0.00 \\ \hline
slk & 57.52 & 18.67 & 63.20 & 2.94 \\ \hline
slv & 55.92 & 18.18 & 65.63 & 0.00 \\ \hline
ukr & 29.56 & 6.45  & 56.83 & 0.00 \\ \hline
all & 49.26 & 18.16 & 64.80 & 1.08 \\ \hline
\end{tabular}
\end{table}

\subsubsection{Liner2}

Авторы модели использовали фреймворк Liner2, который предоставляет набор модулей, основанных на статистических моделях, словарях, правилах и эвристиках и аннотирующих различные типа фраз \cite{marcinczuk-kocon-oleksy:2017:BSNLP}. Команда работала только с польским языком, и сумела достичь лучших результатов в задачах выделения и нормализации именованных сущностей. Результаты работы системы можно увидеть ниже в таблице.

\begin{table}[ht]
\centering
\caption{Liner2 Results}
\label{liner2}
\begin{tabular}{|l|l|l|l|}
\hline
Task                 & P     & R     & F     \\ \hline
Names matching       &       &       &       \\ \hline
Relaxed partial      & 66.24 & 63.27 & 64.72 \\ \hline
Relaxed exact        & 65.40 & 62.78 & 64.07 \\ \hline
Strict               & 71.10 & 58.81 & 66.61 \\ \hline
Normalization        & 75.50 & 44.44 & 55.95 \\ \hline
Coreference          &       &       &       \\ \hline
Document level       & 7.90  & 42.71 & 12.01 \\ \hline
Language level       & 3.70  & 8.00  & 5.05  \\ \hline
Cross-language level & n/a   & n/a   & n/a   \\ \hline
\end{tabular}
\end{table}

\subsubsection{LexiFlexi}

К сожалению, авторы системы не предоставили статьи о своей системе, поэтому приходится довольствоваться ее кратким описанием. LexiFlexi применяет 3 лексико-семантических ресурса ко входному тексту в следующем порядке:

\begin{enumerate}
    \item Сопоставляет имена из базы данных JRC Variant Names 
    \item Сопоставляет имена из огромной коллекции названий сущностей на различных языках полу-автоматически полученной из применения BabelNet к еще не обработанному тексту
    \item Сопоставляет топонимы из газетира GeoNames в необработанном тексте
\end{enumerate}

В конце работы системы применяются несколько языко-независимых эвристик для нахождения вариантов (аббревиатур) именованных сущностей, распознанных на предыдущих шагах.

Данная модель показала средние результаты в задачах выделения и нормализации сущностей, однако заняла практически все первые места в задаче межъязыкового связывания на уровне документов, и половину первых мест - на уровне языка.

\subsubsection{Sharoff}

Система Сергея Шарова \cite{sharoff:2017:BSNLP} - пример применения метода адаптации языка к задаче выделения именованных сущностей. Автор модели создал мультиязычное пространство embedding-ов для слов, основываясь на модели Dinu с добавлением взвешенного расстояния Левенштейна. Это пространство было использовано для обучения NER таггера, созданного при помощи нейронной сети, базируясь на Словенском NER корпусе. Простыми словами, главная идея - если мы можем адаптировать модель отзывов к фильмам для работы с отзывами к отелями, то наверное мы можем адаптировать модель для NER одного языка к работе с родственными языками.

Модель добилась неплохих результатов в задачах выделения и нормализации именных сущностей, заняла первые места на чешском и словенском языках, заметны также и сильные падения качества на русском и украинском. На задаче межъязыкового связывания система проявила себя не очень хорошо, везде осталась на последних местах. 

\subsection{Выводы}

На мой взгляд, на данной конференции были показаны две интересные системы - система Сергея Шарова и система JHU. Система Liner2 была применена только к польскому языку, а система LexiFlexi, если верить ее описанию, просто работает с большим набором газетиров. Результаты, достигнутые системой Шарова и системой JHU, показывают применимость методов адаптации языка и параллельной обработки текстов к задачам NER.
\section{Настройка окружения, пакет NLTK, корпуса}


Исследовательская работа проводилась в ОС Windows 10, использовалась среда для научных исследований Anaconda, Python3. Дополнительно установил пакет pymorphy2 для POS-тегирования токенов из датасетов на русском языке, для английского датасета (CoNLL2003) использовались предоставленные организаторами теги (chunk и POS). Для удобной работы с датасетами они были приведены к модифицированному формату датасетов CoNLL (были добавлены столбцы OFFSET и LEN - отступ токена и его длина), в коде они хранятся в виде модифицированных NLTK Corpus-ов. (код обработки можно найти в файлах .ipynb, а код NLTK Corpus-а - в файле corpus.py)
\section{Выделение признаков}

В первоначальной итерации было решено рассмотреть следующие признаки:
\begin{enumerate}
    \item Часть речи (POS-tag, для CoNLL2003 датасета - и chunk-tag)
    \item Капитализация (normal-case, Proper-case, CAPITAL-case, Camel-case)
    \item Флаг, является ли слово числом
    \item Флаг, является ли слово знаком пунктуации
    \item Начальная форма слова
\end{enumerate}

Для обработки датасетов и генерации признаков был написан отдельный класс Generator (файл generator.py), который на основании входных данных создает матрицу признаков, эта матрица обрабатывается OneHotEncoder-ом из пакета sklearn, опционально сохраняется в файл и возвращается в вызывающий код. Матрица признаков получается очень разреженной, поскольку ее размер напрямую зависит от размера словаря датасета (так, для датасета FactRuEval ее размеры достигают 35.000 * 27.000)
\section{Отбор признаков}

Теперь, когда у нас есть матрица признаков, можно заняться их отбором. Для начала, исключаем самые малоинформативные признаки - те, которые встретились в датасете менее 5 раз. Далее сортируем признаки по весам, присвоенным им классификатором, и отберем признаки, составляющие 90 процентов веса. Отбор признаков происходит при их генерации, код находится в классе Generator. После отбора признаков их число резко снижается - с 27 тысяч до порядка 700 (датасет FactRuEval) в случае 90 процентов веса. Приведем графики качества на тестовой и обучающей выборке в зависимости от оставляемого процента признаков.
\section{Baselines}
\subsection{Получение baseline-ов}

Попробуем обучить основные классификаторы на полученных данных. Рассмотрим такие классификаторы, как GradientBoostingClassifier, RandomForestClassifier, LogisticRegression, LinearSVC, без подбора параметров. Приведем результаты для всех датасетов (в случае BSNLP воспользуемся данными для обучения FactRuEval и приведем предсказанные имена классов к именам BSNLP по правилам LocOrg ${\Rightarrow}$ LOC, Other ${\Rightarrow}$ MISC):


\begin{table}[ht]
\centering
\caption{CoNLL2003 Results}
\label{conll2003}
\begin{tabular}{|l|l|l|l|l|l|}
\hline
            & ORG   & LOC   & MISC  & PER   & Total \\ \hline
LogReg      & 0.760 & 0.820 & 0.787 & 0.888 & 0.826 \\ \hline
RF          & 0.676 & 0.725 & 0.682 & 0.849 & 0.752 \\ \hline
LinSVC      & 0.777 & 0.823 & 0.799 & 0.887 & 0.832 \\ \hline
GB          & 0.677 & 0.729 & 0.748 & 0.834 & 0.758 \\ \hline
\end{tabular}
\end{table}



\begin{table}[ht]
\centering
\caption{FactRuEval Results}
\label{factrueval}
\begin{tabular}{|l|l|l|l|l|l|}
\hline
          & Per    &      Loc & Org   & LO     & Total \\ \hline
LogReg    & 0.802  & 0.551    & 0.460 & 0.555  & 0.577 \\ \hline
RF        & 0.763  & 0.475    & 0.364 & 0.460  & 0.501 \\ \hline
LinSVC    & 0.804  & 0.553    & 0.512 & 0.559  & 0.603 \\ \hline
GB        & 0.759  & 0.493    & 0.413 & 0.509  & 0.530 \\ \hline
\end{tabular}
\end{table}



\begin{table}[ht]
\centering
\caption{BSNLP EU Results}
\label{bsnlp_eu}
\begin{tabular}{|l|l|l|l|l|l|}
\hline
            & ORG   & LOC   & MISC  & PER   & Total \\ \hline
LogReg      & 0.652 & 0.504 & 0.000 & 0.313 & 0.525 \\ \hline
RF          & 0.386 & 0.403 & 0.000 & 0.225 & 0.328 \\ \hline
LinSVC      & 0.668 & 0.540 & 0.000 & 0.359 & 0.543 \\ \hline
GB          & 0.655 & 0.497 & 0.000 & 0.368 & 0.529 \\ \hline
\end{tabular}
\end{table}



\begin{table}[ht]
\centering
\caption{BSNLP Trump Results}
\label{bsnlp_trump}
\begin{tabular}{|l|l|l|l|l|l|}
\hline
            & ORG   & LOC   & MISC  & PER   & Total \\ \hline
LogReg      & 0.426 & 0.820 & 0.000 & 0.883 & 0.756 \\ \hline
RF          & 0.289 & 0.700 & 0.000 & 0.823 & 0.674 \\ \hline
LinSVC      & 0.391 & 0.805 & 0.000 & 0.860 & 0.734 \\ \hline
GB          & 0.285 & 0.782 & 0.000 & 0.840 & 0.703 \\ \hline
\end{tabular}
\end{table}

\subsection{Оценка результатов}

Оценка результатов работает с проверкой полного совпадения предсказанной сущности с истинной, если же есть нарушение - вся сущность не попадает в TP, подобный метод проверки отличается от потокеновой проверки (потокеновая проверка более "мягкая"). Для получения итогового результата (не разбитого по классам) складываем полученные по классам TP, FP, FN и вычисляем F1 меру.
\section{Статьи, рассматривающие "глобальные" признаки}

Приведем краткий обзор статей, в которых так или иначе использовались "нелокальные" (иначе глобальные) признаки. 

Так, в статье "Design Challenges and Misconceptions in Named Entity Recognition" \cite{ratinov-roth:2009:CoNLL} рассказывается про то, как авторы объединили идеи из статей "Named Entity Recognition with a Maximum Entropy Approach" \cite{Chieu:2003:NER:1119176.1119199} (использование так называемых глобальных списков), "An Effective Two-Stage Model for Exploiting Non-Local Dependencies in Named Entity Recognition" \cite{krishnan-manning:2006:COLACL} (главная идея которой была в том, чтобы сначала применить baseline-систему к датасету, а потом использовать результаты как признаки), а также использовали идею "Extended prediction history" (использование истории о предыдущих предсказанных классах для токена на протяжении 1000 токенов).

Авторы статьи сумели добиться хорошего результата в 90.8. 

\iffalse

Рассмотрим подробнее идею глобальных списков из первой статьи. В процессе обработки датасета создаются следующие листы:
\begin{enumerate}
    \item Frequent Word List - слова, встречающиеся более чем в пяти документах.
    \item Useful Unigrams - For each name class, words that precede the name class are ranked using correlation metric (Chieu and Ng, 2002a), and the top 20 are compiled into a list.
    \item Useful Bigrams - (UBI) This list consists of bigrams of words that precede a name class. Examples are “CITY OF”, “ARRIVES IN”, etc. The list is compiled by taking bigrams with higher probability to appear before a name class than the unigram itself (e.g., “CITY OF” has higher probability to appear before a location than “OF”). A list is collected for each name class.
    \item Useful Word Suffixes - (SUF) For each word in a name class, three-letter suffixes with high correlation metric score are collected.
    \item Useful Name Class Suffixes - A suffix list is compiled for each name class. These lists capture tokens that frequently terminate a particular name class.
    \item Function Words - Lower case words that occur within a name class. These include “van der”, “of”, etc.
\end{enumerate}

После чего на основании полученных листов получаются глобальные признаки:

\begin{enumerate}
    \item Unigrams - If another occurrence of w in the same document has a previous word wp that can be found in UNI, then these words are used as features Other- occurrence-prev=wp.
    \item Bigrams - If another occurrence of w has the feature BI-nc set to 1, then w will have the feature OtherBI-nc set to 1.
    \item Class Suffixes - If another occurrence of w has the feature NCS-nc set to 1, then w will have the feature OtherNCS-nc set to 1.
    \item InitCaps of Other Occurrences - This feature checks for whether the first occurrence of the same word in an unambiguous position (non first-words in the TXT zone) in the same document is initCaps or not. For a word whose initCaps might be due to its position rather than its meaning (in headlines, first word of a sentence, etc), the case information of other occurrences might be more accurate than its own.
    \item Name Class of Previous Occurrences - The name class of previous occurrences of w is used as a feature, similar to (Zhou and Su, 2002). We use the occurrence where w is part of the longest name class phrase (name class with the most number of tokens).
\end{enumerate}

\fi
\section{Использование глобальных признаков}

Для начала попробуем использовать так называемую историю предсказаний. (отсылка на статью с историей). Идея проста - при создании признаков для токена будем использовать информацию о предыдущих вхождениях токена (например, на протяжении 1000 последних токенов) - создадим столбцы для этого, или же, согласно статье ..., просто будем хранить вероятностную информацию о предсказаниях.



{\footnotesize 
\bibliographystyle{acm}
\bibliography{biblio}
}

\end{document}
